We have reviewed the concepts, the particular theories and development made in history, and the current usage of the treatment in the field of artificial intelligence and by extension to the modern machine learning theory. While there are much to be desired, it is imperative to note that we did not do a thorough overview, but the general theoretical interest. 

With that, comes evaluation of the classical stance. Even though the classical theory can be seen as quite advanced and matured, in fact, it is not. Rather, it is endowed in the world of mathematics and empiricalism, for its advancements and development as present. Furthermore, which is one of the factor that inflicts the status of \textit{naive} onto the classical treatment, is the overall no coherence structure - given how much and how rapid the changes have been - of the establishment, both practical-wise and theory-wise. Science itself, particularly, also has this problem. Other than that, there is also the fact that the current framework of theoretical artificial intelligence is not matured enough to take on various new questions and problems that are detrimental to their operation and mechanism. If one was to say, then they would be saying that while we dream of AGI (Artificial General Intelligence), we have no tools capable of reaching it aside from the dreams come true of wishes. 

It is again, not superficially mentioned to downplay the impact and role of classical theory in the development and conceptualization of artificial constructs. It is, however, inadequate of the present, insufficient for the future, and inoperable in practice. We then, is inclined to accept and continue on toward a newer approach. 

The following part of several chapters will be dedicated specifically to allow for the perhaps new formalism and theory to grow. In such, we will tackle those problems that the old scientist and researchers failed, those problems that was created from their insight (and failure), and lessons from successes. Fortunately for us, those are plenty, and we have to look not too further from the starting point to see the beacon where we will proceed. 

\chapter{General principles}



For much of the history, as well as the current conception of artificial intelligence designing, the design and implementation of AI has been influenced into the four main paradigm, as mentioned in Norvig's Artificial Intelligence. A modern approach (2011). Alternatively, it also follows the more general view toward either looking specifically at the structural components of any specimen of intelligence (human neurons and animal's), or \textit{neurological approach}, also called \textit{connectionism}; and those that follow the view of universal rationalism, capturing the essence behavioural constructs, and basing AI constructions on such universal justification, the \textit{symbolic AI} approach. While those theories can be said to eventually be useful and perhaps, more groundbreaking than not, they are not so much desirable, as they are also very well-being specialized, unscalable, and was stuck in the paradigm of the more simpler tasks. In fact, whilst NLP, the chatbot that is prominent in the later date of the 2010s was hailed to be the epitome of development in AI (which indeed, perhaps it is), the issues with, for example, *common sense knowledge* is still present, and any attempt in fixing it often resulted in either the rough, uncertain and naive mimic of the 'real thing', or fails miserably. In one way or another of such, we can say that the current AI theory has no \textit{framework}. Or at least a manageable one. 

It is then of our interest to put up at least a view of what should be, and what can be said of artificial intelligence as a framework, a system for theories, and the principles in which one can say about the core essence of artificial intelligence, given an interpretation. In said following sections, we would allow ourselves to tread the thin line, and lay out the foundation of what to develop to a full-fledged artificial intelligence theory. 

What should be expected of this outline? Not so much. The ultimate goal here is to formalize the principles, the working mechanism, the designing compass in which developments might ensue of a later date. Settle on matters that of said digression, will influence how the entire framework is formed, from the macroscopic view to the microscopic design elements, the point of view of the system, and more on the treatment of such system in certainty. Or probabilistically, which you can choose. In fact, the issues between determinism and uncertainty (probability interpretation) will also be touched upon, forgoes the huge amounts of works on two said interpretations ambiguous as it is possible to be. 
\section{Defining artificial intelligence}
There are many ways to define artificial intelligence, either by the phenomenological one, or by the presumptuous, universal-assumed way of logicians in the old way. However, our definition, or at least mechanism of working on artificial intelligence, should be by then very different, as to not stumble upon the mistake of overestimating our predicate, and neglect the hardness of the problem itself.

What is then called AI, in our view? We treat AI as rather a more general system. First, we digress on the term \textbf{artificial}. We have talked about this in previous chapters, but, for the moment, it is perhaps better to reformulate it. A construct, object, subject $A$ is called \textit{artificial}, if it fits the following rough conceptual definition. 
\begin{definition}[Artificial]
    We say an object $A$ is artificial only if it is not natural, or rather, it is \textit{intentionally created} by meaning intentions, and not the general evolution of states, the natural interaction of the law of nature, or the natural transformation of biology.
\end{definition}
By such definition, artificial is then a quality to be separated from \textit{natural intelligence} - of naturally made intelligent vessels or construct, existed because of natural, biological evolution and advancements by itself. A construct, then, is the main interest of our study. What about intelligence then, one might ask? The truth is, we don't know. We know roughly that intelligence is the exhibition of rationalism, the actions and demonstration of perhaps consciousness, of thinking. However, we have no way to define operationally, formally, and if not, conceptually sound of such aspect of intelligence. Let's assume we don't know. Then artificial intelligence is especially the realization and the discovery of intelligence itself. This is pretty much universal as we can get, out of the existing structure and predicament. Overall, the \textit{universal argument} that would be utilized in developing our conceptual shell of artificial intelligence, shall not be too restrictive as for the universality of formal logic, for the world unable to be fully realized in formalism terms. We are now ready for a conceptual definition of artificial intelligence. 
\begin{definition}[Artificial intelligence]
    A \textbf{construct} $U$, subjected to a system $S$ is called \textbf{artificial intelligence} if it satisfies the condition of being artificial, whilst also satisfies a given criterion set of being autonomous, dynamic, and overall general. It is such that will give rise to \textit{intelligence construct}. 
\end{definition}

While it is dubious, we will further develop those points made above. However, we might as well want to justify the first point in all - the point of the intelligence criterion. 
\subsection{Intelligence criterion}

% PACKAGE INCLUSION
\usepackage{amsbsy,amssymb,amsmath,amsfonts,amsthm} % Math and related rendering
\allowdisplaybreaks
\usepackage{tikz-cd,tikz-3dplot,circuitikz} % For graphical drawing
%\usepackage{natbib} % For citation, using numbering style. 
\usepackage{graphicx}
\usepackage{geometry} % configure the geometry page
\usepackage{bbm} % Better bold (for example, \mathbb{1} does not work for font not alphabetical)
\usepackage{bm} % Bold symbol 
\usepackage{tikz} % for normal graphing
\usepackage{enumitem} % for customizing options for lists, enumerations, and descriptions
\usepackage{fancyhdr} % configuring header
\usepackage{parskip} % For paragraph spacing
\usepackage{caption} % captioning options
\usepackage{mathtools} % using this to fix \multlined environment altogether
\usepackage{parnotes}% take this one for marginnote substitute (or the reverse because this one looks so much better)
\usepackage{braket} % For bra-ket notation
%\usepackage{fontspec} % does fontspec works for memoir? 
\usepackage{float} % more float control
\usepackage[dvipsnames,svgnames,x11names]{xcolor} % color schema, definitely should have included this
\usetikzlibrary{matrix}
\usepackage{thmtools} % hooks and stuff for listoftheorem
%\DisemulatePackage{showidx} %(2)
%\usepackage{makeidx} % I HATE THIS
%\usepackage{showidx} % I HATE THIS
%\usepackage{subfigure}
\usepackage{caption} % uh, caption
\usepackage{nicematrix} % for drawing matrix
\usepackage{subcaption} % This and the two right above it is required for multi-figure display, so yeah.
\setcounter{tocdepth}{2}
%-------------------------------%
% CONFIGURING NATBIB
\usepackage{natbib} 
%-------------------------------%
% NEXT IS INDEX CONFIG
%%%%%%%%%%%%%%%%%%%%%%%%%%%
\usepackage{hyperref} % For referencing
\makeatletter
\renewcommand{\index}[1]{%
  \oldindex{#1}%
  \if@reversemargin
    \marginpar{\raggedleft\small#1}%
  \else
    \marginpar{\raggedright\small#1}%
  \fi
}
\makeatother
%\showindexmarks
\makeindex
\let\oldindex\index
\renewcommand{\index}[1]{\oldindex{#1}\marginpar{\hspace{3mm}\small#1}}
%\newcommand\boldblue[1]{\textcolor{blue}{\textbf{#1}}}
%####################################
\let\oldtextit\textit 
\renewcommand\textit[1]{\oldtextit{\color{orange!70!black}#1}}
%\let\oldtextbf\textbf
%\renewcommand\textbf[1]{\oldtextbf{\color{orange!90!black}#1}}
%####################################
\setlength{\parskip}{3pt}
\let\MakeUppercase\relax
\fancyhfinit{\sffamily\bfseries}
%%%%%%%%%%%%%%%%%%%%%%%%%%%
% Titlepage configuration
\newcommand{\HRule}[1]{\rule{\linewidth}{#1}} 	% Horizontal rule

\makeatletter							% Title
\def\printtitle{%						
    {\centering \@title\par}}
\makeatother									

\makeatletter							% Author
\def\printauthor{%					
    {\centering \large \@author}}				
\makeatother

\title{	\normalsize \textsc{Lecture notes} 	% Subtitle
		 	\\[3.0cm]								% 2cm spacing
			\HRule{0.5pt} \\						% Upper rule
			\huge {\sffamily The Physics Compendium}% Title
      
			\HRule{1pt} \\ [2.5cm]		% Lower rule + 0.5cm spacing
			\normalsize	\normalfont 		% Todays date
		}
\author{
		F. Amane\\ Department of Physics
}
\date{Last comprehensive review: May 18, 2025}
% End config
%%%%%%%%%%%%%%%%%%%%%%%%%%%
% MARGIN
%\setstocksize{31.2cm}{22.5cm}
\setlrmarginsandblock{4.8cm}{2.5cm}{1}
%\setulmarginsandblock{2.5cm}{*}{1}
\checkandfixthelayout 
%%%%%%%%%%%%%%%%%%%%%%%%%%%%

%%%%%%%%%%%%%%%%%%%%%%%%%%%%%%%%%%%%%%%%%%%%%%%%%%%%%%% COLOR BOX CONFIG
\theoremstyle{plain}
\newtheorem{definition}{Definition}[section]
\newtheorem{theorem}{Theorem}[section]
\newtheorem{col}{Corollary}[section]
\newtheorem{conjecture}{Conjecture}[section]
\newtheorem{setting}{Setting}[section]
\newtheorem{proposition}[theorem]{Proposition}
\newtheorem{lemma}[theorem]{Lemma}
\newtheorem{assumption}[theorem]{Assumption}
\newtheorem{assume}{Assumption}[subsection]
\newtheorem{remark}[theorem]{Remark}
\newtheorem{hypothesis}{Hypothesis}[section]
\newtheorem{axiom}{Axiom}[section]
\newtheorem{question}{Question}[section]
\newtheorem{example}{Example}[section]
\newtheorem{note}{Note}[section]
%%%%%%%%%%%%%%%%%%%%%%%%%%%%
%\makeatletter
%\renewcommand{\sidepar}[1]{\marginpar{\footnotesize #1}}
%\makeatother
%%%%%%%%%%%%%%%%%%%%%%%%%%%%
\usepackage[breakable,theorems,skins]{tcolorbox}
\tcbuselibrary{breakable}
\tcbset{%
  breakable,
  enhanced jigsaw,
  parbox=false,
  nobeforeafter,
  before skip=10pt,
  after skip=10pt,
  arc=2pt,
}
\tcolorboxenvironment{theorem}{
  breakable,
  colback=blue!5!white,
  boxrule=0pt,
  boxsep=1pt,
  left=2pt,right=2pt,top=2pt,bottom=2pt,
  oversize=2pt,
  before skip=\topsep,
  after skip=\topsep,
  enhanced jigsaw,
}
\tcolorboxenvironment{definition}{
  colback=gray!10!white,
  boxrule=0pt,
  boxsep=1pt,
  left=2pt, right=2pt, top=2pt, bottom=2pt,
  oversize=2pt,
  %sharp corners,
  before skip=\topsep,
  after skip=\topsep,
}
\tcolorboxenvironment{col}{
  colback=orange!10!white,
  boxrule=0pt,
  boxsep=1pt,
  left=2pt, right=2pt, top=2pt, bottom=2pt,
  oversize=2pt,
  %sharp corners,
  before skip=\topsep,
  after skip=\topsep,
}

\tcolorboxenvironment{conjecture}{
  colback=red!10!white,
  boxrule=0pt,
  boxsep=1pt,
  left=2pt, right=2pt, top=2pt, bottom=2pt,
  oversize=2pt,
  %sharp corners,
  before skip=\topsep,
  after skip=\topsep,
}

\tcolorboxenvironment{setting}{
  colback=blue!40!orange!20!white, % Mix blue (40%) and orange (20%)
  boxrule=0pt,
  boxsep=1pt,
  left=2pt, right=2pt, top=2pt, bottom=2pt,
  oversize=2pt,
  %sharp corners,
  before skip=\topsep,
  after skip=\topsep,
}
% Proposition (green)
\tcolorboxenvironment{proposition}{
  colback=green!5!white,
  boxrule=0pt,
  boxsep=1pt,
  left=2pt, right=2pt, top=2pt, bottom=2pt,
  oversize=2pt,
  %sharp corners,
  before skip=\topsep,
  after skip=\topsep,
}

% Lemma (orange -- toned down)
\tcolorboxenvironment{lemma}{
  colback=orange!10!white,
  boxrule=0pt,
  boxsep=1pt,
  left=2pt, right=2pt, top=2pt, bottom=2pt,
  oversize=2pt,
  %sharp corners,
  before skip=\topsep,
  after skip=\topsep,
}

% Assumption (purple)
\tcolorboxenvironment{assumption}{
  colback=purple!5!white,
  boxrule=0pt,
  boxsep=1pt,
  left=2pt, right=2pt, top=2pt, bottom=2pt,
  oversize=2pt,
  %sharp corners,
  before skip=\topsep,
  after skip=\topsep,
}

% Assume (a variant of Assumption with a slightly different shade)
\tcolorboxenvironment{assume}{
  colback=purple!10!white,
  boxrule=0pt,
  boxsep=1pt,
  left=2pt, right=2pt, top=2pt, bottom=2pt,
  oversize=2pt,
  %sharp corners,
  before skip=\topsep,
  after skip=\topsep,
}

% Remark (gray, very neutral)
\tcolorboxenvironment{remark}{
  colback=gray!5!white,
  boxrule=0pt,
  boxsep=1pt,
  left=2pt, right=2pt, top=2pt, bottom=2pt,
  oversize=2pt,
  %sharp corners,
  before skip=\topsep,
  after skip=\topsep,
}

% Hypothesis (cyan)
\tcolorboxenvironment{hypothesis}{
  colback=cyan!10!white,
  boxrule=0pt,
  boxsep=1pt,
  left=2pt, right=2pt, top=2pt, bottom=2pt,
  oversize=2pt,
  %sharp corners,
  before skip=\topsep,
  after skip=\topsep,
}

% Axiom (yellow -- with extra white to reduce brightness)
\tcolorboxenvironment{axiom}{
  colback=yellow!10!white,
  boxrule=0pt,
  boxsep=1pt,
  left=2pt, right=2pt, top=2pt, bottom=2pt,
  oversize=2pt,
  %sharp corners,
  before skip=\topsep,
  after skip=\topsep,
}

% Question (magenta -- toned down)
\tcolorboxenvironment{question}{
  colback=magenta!10!white,
  boxrule=0pt,
  boxsep=1pt,
  left=2pt, right=2pt, top=2pt, bottom=2pt,
  oversize=2pt,
  %sharp corners,
  before skip=\topsep,
  after skip=\topsep,
}

% Interject (red -- toned down)
\tcolorboxenvironment{interject}{
  colback=red!10!white,
  boxrule=0pt,
  boxsep=1pt,
  left=2pt, right=2pt, top=2pt, bottom=2pt,
  oversize=2pt,
  %sharp corners,
  before skip=\topsep,
  after skip=\topsep,
}

% Example (teal)
\tcolorboxenvironment{note}{
  colback=teal!10!white,
  boxrule=0pt,
  boxsep=1pt,
  left=2pt, right=2pt, top=2pt, bottom=2pt,
  oversize=2pt,
  %sharp corners,
  before skip=\topsep,
  after skip=\topsep,
  enhanced jigsaw,
}
%-----------------------------%
% Setting margin
%\setlength{\textwidth}{5in}
%\setlength{\oddsidemargin}{7pt}
%\setlength{\evensidemargin}{7pt}
%\chapterstyle{default}
%%\renewcommand{\chapternamenum}{}
%\setlength{\beforechapskip}{0pt}
%\setlength{\midchapskip}{0pt}
%\setlength{\afterchapskip}{0pt}
%\renewcommand*{\chapnumfont}{\normalfont\Huge\bfseries\sffamily}
%\renewcommand*{\chaptitlefont}{\normalfont\Huge\bfseries\sffamily}

%\makechapterstyle{sfspaced}{
%  \renewcommand{\chapnamefont}{\sffamily\bfseries\LARGE}
%  \renewcommand{\chapnumfont}{\sffamily\bfseries\LARGE}
%  \renewcommand{\chaptitlefont}{\sffamily\bfseries\Huge}
%
%  \renewcommand{\printchaptername}{\chapnamefont Chapter}
%  \renewcommand{\printchapternum}{\chapnumfont\thechapter}
%  \renewcommand{\afterchapternum}{\par\vspace{1em}} % vertical space
%  \renewcommand{\printchaptertitle}[1]{\chaptitlefont ##1}
%}
%
%\chapterstyle{sfspaced}
%\let\chaptername\relax
%\makeatletter
%\makechapterstyle{thesis}{
%\renewcommand{\chapternamenum}{}
%\setlength{\beforechapskip}{0pt}
%\setlength{\midchapskip}{0pt}
%\setlength{\afterchapskip}{0pt}
%\renewcommand{\chapnamefont}{\LARGE}
%\renewcommand{\chapnumfont}{\chapnamefont}
%\renewcommand{\chaptitlefont}{\chapnamefont}
%\renewcommand{\printchapternum}{}
%\renewcommand{\afterchapternum}{}
%\renewcommand{\printchaptername}{}
%\renewcommand{\afterchaptertitle}{\chapnumfont\hfill\thechapter\\\vspace*{-.3cm}%\hrulefill\vspace*{6cm}\\}
%}
%\makeatother
%\setbeforesecskip{0pt}
%\setaftersecskip{0pt}
%\setbeforesubsecskip{0.5pt}
%\setaftersubsecskip{0.25pt}
\setlength{\itemindent}{-2em}
\titleformat{\chapter}{\huge\sffamily}{\chaptername~\thechapter.}{1em}{}{}
%\titleformat{\section}{\large}{\thesection}{1em}{}{}
%\titleformat{\subsection}{\normalsize}{\thesubsection}{1em}{}{}
\setsecnumdepth{subsection}
\setsecheadstyle{\sffamily\Large}
\setsubsecheadstyle{\sffamily\large}
\setsubsubsecheadstyle{\sffamily\normalsize}
\setparaheadstyle{\sffamily\normalsize\itshape}
% FOOTNOTE PENALTY
\interfootnotelinepenalty=9000
\titlespacing{\section}{0pt}{5pt}{3pt}
\titlespacing{\subsection}{0pt}{3pt}{2pt}
\titlespacing{\subsubsection}{0pt}{1pt}{1pt}

\setlength{\textfloatsep}{10pt plus 1.0pt minus 2.0pt}

%----------------------------------------------------------------------------------% 
% SMALL SETTING
\usepackage{pgfplots} % for something else
\usetikzlibrary{patterns}       % For custom patterns
\usetikzlibrary{shadings}       % For gradient fills
\usetikzlibrary{shapes.geometric} % For geometric shapes
\usetikzlibrary{calc}           % For coordinate calculations
\usetikzlibrary{positioning}    % For node positioning
\usetikzlibrary{decorations.pathreplacing} % For path decorations
\usetikzlibrary{fit} % For fitting shapes around nodes
\newcommand{\calx}[1]{\mathcal{#1}}
\newcommand{\mynote}[1]{\medskip\par\textbf{\small Note}\quad\setlength{\extrarowheight}{2pt}\begin{tabularx}{\linegoal}{X}
\Xhline{1pt}
\rowcolor{WhiteSmoke!80!Lavender}#1 \\
\Xhline{1pt}
\end{tabularx}}

\makeatletter
\pgfutil@ifundefined{pgf@pattern@name@_xg1qse1zm}{
  \pgfdeclarepatternformonly[\mcThickness,\mcSize]{_xg1qse1zm}
  {\pgfqpoint{0pt}{-\mcThickness}}
  {\pgfpoint{\mcSize}{\mcSize}}
  {\pgfpoint{\mcSize}{\mcSize}}
  {
    \pgfsetcolor{\tikz@pattern@color}
    \pgfsetlinewidth{\mcThickness}
    \pgfpathmoveto{\pgfqpoint{0pt}{\mcSize}}
    \pgfpathlineto{\pgfpoint{\mcSize+\mcThickness}{-\mcThickness}}
    \pgfusepath{stroke}
  }
}
\makeatother

\makeatletter
\pgfutil@ifundefined{pgf@pattern@name@_lbrsyyeax}{
  \pgfdeclarepatternformonly[\mcThickness,\mcSize]{_lbrsyyeax}
  {\pgfqpoint{0pt}{0pt}}
  {\pgfpoint{\mcSize+\mcThickness}{\mcSize+\mcThickness}}
  {\pgfpoint{\mcSize}{\mcSize}}
  {
    \pgfsetcolor{\tikz@pattern@color}
    \pgfsetlinewidth{\mcThickness}
    \pgfpathmoveto{\pgfqpoint{0pt}{0pt}}
    \pgfpathlineto{\pgfpoint{\mcSize+\mcThickness}{\mcSize+\mcThickness}}
    \pgfusepath{stroke}
  }
}
\makeatother
%-----------------------------------------------------------------------------------------%
%\renewcommand\index{\@bsphack\begingroup\@sanitize\catcode32=10\relax\@index}
%
%\renewcommand\makeindex{\if@filesw \newwrite\@indexfile
%\immediate\openout\@indexfile=\jobname.idx
%\def\index{\@bsphack\begingroup
%\def\protect####1{\string####1\space}\@sanitize
%\catcode32=10 \@wrindex\@indexfile}\typeout
%{Writing index file \jobname.idx }\fi}
%
%\def\@wrindex#1#2{\let\thepage\relax
%\xdef\@gtempa{\write#1{\string
%\indexentry{#2}{\thepage}}}\endgroup\@gtempa
%\@showidx{#2}\if@nobreak \ifvmode\nobreak\fi\fi\@esphack}
%------------------------------------------------------------------------------------------%
% Test blockquote

\usepackage{csquotes}
\makeatletter
\patchcmd{\csq@bquote@i}{{#6}}{{\emph{#6}}}{}{}
\makeatother
\renewcommand{\mkbegdispquote}[2]{\itshape}
\newcommand{\cH}{{\mathcal{H}}}
\newcommand{\cHN}{{\mathcal{H}_N}}
\newcommand{\rkhs}{{\mathcal{H_\infty}}}

% Reference box setup
\hypersetup{
    pagebackref=true,
    hyperindex=true,
    colorlinks=true,
    breaklinks=true,
    urlcolor=black,
    filecolor=black,
    citecolor=blue,
    linkcolor=black, % <--- prevents ToC and section links from being colored
    bookmarks=true,
    bookmarksopen=false
}


\ifodd\value{page}
  \normalmarginpar
\else
  \reversemarginpar
\fi

\newcommand{\x}{\times}
\newcommand{\cD}{\mathcal{D}}
\newcommand{\cX}{\mathcal{X}}
\newcommand{\cY}{\mathcal{Y}}
\newcommand{\cL}{\mathcal{L}}
\newcommand{\cA}{\mathcal{A}}
\newcommand{\cT}{\mathcal{T}}

\chapter{Introduction}

This book serves a dual purpose - one is to record my understanding and knowledge about the topic, and related to it, and one is to develop my own theories and models - most likely original, more often than not it's not. Also, it will also include rough implementation, papers' ideas analysis, and else. So, a bag full of everything, I suppose.  a dual purpose - one is to record my understanding and knowledge about the topic, and related to it, and one to directly record and write down my discovery, or rather, most of the time just analysis and designs. 

It might sound as if this book is entirely self-sufficient, and also self-serving, yet it isn't. The text cannot take into account all preliminaries or requirements of understanding, yet. It is simply too large to take all in, because the field itself is very complex. Self-serving side, perhaps it can be broken down to $0-1$ loss function two (it's a joke), since it is perhaps too difficult to follow directly from the get-go, so I pretend to be the third-party narrative. With that, the book is not just about myself asking and answering questions, logging knowledge, but to also write and try to explain it as a perhaps to a 5-year-old, to somehow make it sounds not like listening to quantum mechanics. 

Overall, I seriously hope that this book will help me myself, ultimately, and help anyone along the way, if they stumble upon this. Hopefully you will give me some exposure too (famous?), so, I am counting on you. With that said, artificial intelligence is a hoax. Only us can refute that. 

\section{What is in the book}
The book is concerned of the main umbrella topic of \textbf{artificial intelligences}. Alongside with it are relevant theories and practical implementations that support said theoretical view. So, you would also expect to have \textit{machine learning, mathematical modelling, a lot of mathematics, information theory, complexity analysis}, and more. Specifically, there will also be an entire large chapter on the formalism of the learning theory, in a rigorous sense, so many of the chapters would be there to reinforce this. 

As it currently stands, the main top category is the parts. Specifically, there are three main parts. Part I on \textit{theory} - the supporting theory, discussions, results and analysis. Part II on \textit{advanced theory}, being called advanced just because it is my own implementation and theory in accordance. And part III on \textit{implementation} and any necessary details on such - so, it can contain sections on the Python language itself (which is boring) - but also sections concerning deployment or rather typical construction of what has been established in the preceding part. Though there are plans to go for C++ implementation, high-intensity computation is perhaps not in the list for the current time (April 2024). 
\section{How much topic is covered?}
Well, not so much, but I hope it will be enough to formalize and construct a formal treatment of a potentially modern approach to artificial intelligence theory. Of the latest revision, however (May 18, 2025), the book has been subsequently changed of its status as a comprehensive AI-based book, to a rather general book by its own standard. Though, it does not contrive the structure for literally everything, so mostly we will narrow down to either philosophy, computer science theory, artificial intelligence, mathematics, and physics. 